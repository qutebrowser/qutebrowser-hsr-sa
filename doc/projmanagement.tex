% -*- TeX-master: "main.tex" -*-

% Projektmanagement (Planung, Soll)
\chapter{Project Management}
\label{ch:projectman}

% Prototypen, Releases, Meilensteine
\section{Prototypes, Releases, Milestones}
\fixme{Split up actual releases?}

Since this project is done without any external industry partner, there is no
immediate need for fixed releases or prototypes during the SA.

However, the following releases were planned and shipped:

\begin{itemize}
  \item A v1.5.0 feature release as soon as PyQt 5.11.3 is released upstream:
    \begin{itemize}
      \item v1.5.0 was released in week 3.
    \end{itemize}
  \item Patch releases for v1.5.x as needed in case of regressions or serious
    enough bugs:
    \begin{itemize}
      \item v1.5.1 was released in week 4;
      \item v1.5.2 in week 6;
      \item No further patch releases were needed.
    \end{itemize}
  \item v1.6.0 after the work on refactoring qutebrowser's core into extensions
    is completed, in order to get the changes out to users as soon as possible.
    \begin{itemize}
      \item It was decided to delay the v1.6.0 release to January, synchronizing
        it with the release of PyQt
        5.12\footnote{\url{https://www.riverbankcomputing.com/pipermail/pyqt/2018-December/041192.html}}.
    \end{itemize}
  \item v1.6.x patch releases as needed (probably after the SA is finished).
  \item In the future, after the extension API is open for third-party
    extensions, a v2.0.0 release with some other big changes (like dropping support
    for the older QtWebKit backend, as it's rarely updated upstream, even for
    security issues).
\end{itemize}

At the end of week 11, an API review milestone is planned, where the extension
API design gets reviewed by employees of the IFS (Institute for Software) at the
HSR.

% Team, Rollen und Verantwortlichkeiten
\section{Team and Roles}
\label{sec:team}
\emph{Florian Bruhin} is both the primary maintainer of qutebrowser and the
sole author of this student research project. He has been working on qutebrowser since
December 2013 and started studying Computer Science at HSR in 2016.

\emph{Joël Schwab} intended to co-author this research project, but
unfortunately didn't pass some exams, which were a precondition to be allowed to
do the SA project this semester.

Professor \emph{Stefan Keller}, institute partner at the Institute for Software
(IFS) at HSR is the advisor for this project.

\emph{Raphael Das Gupta}, research assistant at the IFS, helped out with code
reviews and architectural decisions.

The qutebrowser community is not directly involved in this research project, but
is the primary audience of the resulting work. It has also contributed many
ideas and use cases for future
extensions\footnote{\url{https://github.com/qutebrowser/qutebrowser/issues/30}}.
There is no external industry partner.

\label{fiete}
\emph{Fritz Reichwald} (fiete201\footnote{\url{https://github.com/fiete201}})
is a long-time qutebrowser user who started working on migrating qutebrowser's
documentation system from asciidoc to Sphinx (see section
\ref{sec:goals-refactorings}). However, he was unable to finish his work in
time, so all API documentation work mentioned in this documentation is the
author's own.

% Prozessmodel
\section{Process Model}
Because of previous experience as part of HSR's ``Engineering Project'', the
``Scrum+'' process model will be used, i.e., an agile Scrum process with an
additional ``End of Elaboration'' milestone.

At the beginning of the SA, there was a phase with the goal of merging existing
third-party contributions (pull requests) into qutebrowser, so that the changes
required for extensions does not conflict with existing work. Thus, an additional
\emph{preparation} phase has been introduced, resulting in the following phases:

\begin{itemize}[parsep=5pt]
  \item Inception
  \item Preparation
  \item Elaboration
  \item Construction
  \item Transition
\end{itemize}

% Aufwandschätzung, Zeitplan, Projektplan
\section{Project Schedule}
\label{schedule}

8 ECTS credits are awarded for the SA-course. Since one ECTS point is equivalent
to a workload of 25--30 hours \autocite{ects} and this SA is written by a single
author, this amounts to 200--240 hours of work. The semester consists of 14
weeks, thus, an average workload of 14--17 hours per week is expected. This time
is allotted as shown in figure \ref{img:schedule}.

Due to the relatively long \emph{preparation} phase at the beginning of the
project, the \emph{construction} phase gets unusually short. This is
unfortunate but unavoidable -- it's important to integrate existing
contributions before starting to refactor the existing code, at least for
third-party changes which would lead to conflicts with refactoring changes.

\begin{figure}[p]
  \begin{ganttchart}[
        title height = 1,
        y unit title=.5cm,
        y unit chart=.5cm,
        bar/.append style={fill=green!20},
        vgrid={
          *1{blue!60},
          *4{gray, dashed},
          *1{blue!60},
          *2{gray, dashed},
          *1{blue!60},
          *2{gray, dashed},
          *1{gray, dashed}
        }]{1}{14}
    \gantttitle{\hspace{-6.5em}\footnotesize{Calendar week}}{0}{0}
    \gantttitlelist{38,...,51}{1} \\
    \gantttitle{\hspace{-6.5em}\footnotesize{Semester week}}{0}{0}
    \gantttitlelist{1,...,14}{1} \\

    \ganttgroup{Inception}{1}{1} \\
    \ganttmilestone{M0: Kickoff}{1} \\

    \ganttgroup{Preparation}{2}{6} \\
    \ganttbar{Qt 5.12 release}{2}{2} \\
    \ganttbar{Merge contributions}{3}{5} \\
    \ganttbar{Buffer}{6}{6} \\

    \ganttmilestone{M1: Most contributions addressed}{6} \\

    \ganttgroup{Elaboration}{7}{9} \\
    \ganttbar{Existing approaches}{7}{7} \\
    \ganttbar{Project documentation}{8}{8} \\
    \ganttbar{API design}{9}{9} \\

    \ganttgroup{Construction}{10}{13} \\
    \ganttbar{Introducing mypy}{10}{10} \\
    \ganttbar{Adding extension API}{11}{12} \\
    \ganttmilestone{M2: API review}{11} \\
    \ganttbar{Refactoring core into extensions}{11}{12} \\
    \ganttmilestone{M3: Feature freeze}{12} \\
    \ganttbar{Buffer}{13}{13} \\

    \ganttgroup{Transition}{14}{14} \\
    \ganttbar{Finishing documentation}{14}{14} \\

    \ganttmilestone{M4: Project end}{14}
  \end{ganttchart}
  \caption{Project schedule}
  \label{img:schedule}
\end{figure}

% Risiken
\section{Risks}
\label{sec:risks}
The following risks have been identified at the beginning of this project:

\definecolor{risk9}{RGB}{165,21,38}
\definecolor{risk8}{RGB}{221,64,45}
\definecolor{risk7}{RGB}{249,141,82}
\definecolor{risk6}{RGB}{254,208,130}
\definecolor{risk5}{RGB}{244,242,162}
\definecolor{risk4}{RGB}{202,233,134}
\definecolor{risk3}{RGB}{133,202,104}
\definecolor{risk2}{RGB}{47,160,83}
\definecolor{risk1}{RGB}{0,104,55}

\begin{figure}[H]
  \begin{tikzpicture}[
    table/.style={
      matrix of nodes,
      row sep=0,
      column sep=0,
      nodes={rectangle,text width=6.4em,align=center},
      text depth=1.5em,
      text height=1em,
      nodes in empty cells
    },
    %
    column 1/.style={nodes={text width=3ex}},
    row 6/.style={nodes={text height=2ex}},
    row 7/.style={nodes={text height=2ex}},
    %
    row 1 column 6/.style={nodes={fill=risk9}},
    %
    row 1 column 5/.style={nodes={fill=risk8}},
    row 2 column 6/.style={nodes={fill=risk8}},
    %
    row 1 column 4/.style={nodes={fill=risk7}},
    row 2 column 5/.style={nodes={fill=risk7}},
    row 3 column 6/.style={nodes={fill=risk7}},
    %
    row 1 column 3/.style={nodes={fill=risk6}},
    row 2 column 4/.style={nodes={fill=risk6}},
    row 3 column 5/.style={nodes={fill=risk6}},
    row 4 column 6/.style={nodes={fill=risk6}},
    %
    row 1 column 2/.style={nodes={fill=risk5}},
    row 2 column 3/.style={nodes={fill=risk5}},
    row 3 column 4/.style={nodes={fill=risk5}},
    row 4 column 5/.style={nodes={fill=risk5}},
    row 5 column 6/.style={nodes={fill=risk5}},
    %
    row 2 column 2/.style={nodes={fill=risk4}},
    row 3 column 3/.style={nodes={fill=risk4}},
    row 4 column 4/.style={nodes={fill=risk4}},
    row 5 column 5/.style={nodes={fill=risk4}},
    %
    row 3 column 2/.style={nodes={fill=risk3}},
    row 4 column 3/.style={nodes={fill=risk3}},
    row 5 column 4/.style={nodes={fill=risk3}},
    %
    row 4 column 2/.style={nodes={fill=risk2}},
    row 5 column 3/.style={nodes={fill=risk2}},
    %
    row 5 column 2/.style={nodes={fill=risk1}},
  ]
    \matrix (mat) [table]
    {
      5 & & & \RNum{1}: Too many contributions & & \\
      4 & \RNum{2}: Migrating docs not done & & & \RNum{3}: Little time for construction & \\
      3 & \RNum{4}: Breaking API changes & \RNum{5}: Sickness of author & & & \\
      2 & & & & & \\
      1 & & & & & \\
        & 1 & 2 & 3 & 4 & 5 \\
    };
    
    \path
      ([yshift=-20pt]mat-5-2.south west) --
      ([yshift=-20pt]mat-5-6.south east)
      node[below, midway]{Severity};
    \path
      ([xshift=-10pt]mat-1-1.north west) --
      ([xshift=-10pt]mat-4-1.south west)
      node[left, midway, rotate=90]{Likelihood};
  \end{tikzpicture}
  \caption{Risk matrix}
  \label{img:risiko}
\end{figure}

The following measures have been taken to mitigate those risks:

\begin{table}[H]
  \begin{tabulary}{\linewidth}{lLL}
    \toprule
    Risk & Description & Mitigation \\
    \midrule
    \RNum{1} & Too many external contributions & Asking contributors to hold back further contributions; ignoring existing contributions  \\
    \hline
    \RNum{2} & Migrating documentation toolchain to Sphinx (done by external contributor)
    is not done in time & Initial extension API documentation can be separate from
    qutebrowser documentation \\
    \hline
    \RNum{3} & Little time in construction phase & Work on extension API is very
    scalable, buffers in schedule \\
    \hline
    \RNum{4} & Future breaking API changes needed & API is not exposed for third-party
    contributions yet \\
    \hline
    \RNum{5} & Sickness of author & Work on extension API is very scalable \\
    \bottomrule
  \end{tabulary}
  \caption{Risk mitigations}
\end{table}

Since some of these risks are hard to quantify in hours lost (such as ``too many
external contributions''), a probability between 1 and 5 has been assigned instead.

As explained in section \ref{context}, it's expected that third-party
contributors continue to submit changes (in the form of pull requests) while
this SA is ongoing. An
announcement\footnote{\url{https://lists.schokokeks.org/pipermail/qutebrowser-announce/2018-October/000053.html},
  accessed 2018-11-12} was sent out, asking contributors to cut back external
contributions. However, a similar announcement was sent out during past exam
seasons, with mixed results -- many contributors continue to submit changes
regardless. Initially, there's a phase focused on merging existing
contributions (see the project schedule in section \ref{schedule}). Afterwards,
contributions are dealt with on a best effort basis, being ignored if they
are non-trivial.

There's little time in the construction phase compared to other projects.
Additionally, this SA is written by a single author. Thus, sickness or other
obstructions have a bigger than usual impact. This is counteracted by trying to
keep the scope of the SA clear, but flexible. Additionally, buffer time is
added after the preparation and construction phases in the project schedule
(section \ref{schedule}).

If some preparatory work isn't finished fully, it's possible that upcoming
changes cause incompatible changes in the extension API. However, this SA
is focused on moving core code into extensions, rather than providing an
extension API for third-party extensions. Therefore, it's still possible to make
those changes after the SA is finished, without breaking third-party code.