% -*- TeX-master: "main.tex" -*-

\chapter{Implementation and Test}
\label{ch:implementation}

\section{Type checking}
\label{sec:mypy}
\subsection{Background}
When it comes to type systems, programming languages are usually categorized
using two properties: Being either \emph{dynamically} or \emph{statically}
typed, and being either \emph{weakly} or \emph{strongly} typed.

Weak and strong typing describe how a language reacts when incompatible types
are combined. Weakly typed languages (such as JavaScript) convert between types
implicitly. In other words, in an expression such as
\js{3 - '1'}, the string \js{'1'} is converted to a number (since it should be
subtracted from another number), with the expression resulting in the number
\js{2}.

Python is strongly typed, therefore, the same expression in Python raises an
exception:

\begin{minted}{pycon}
>​>​> 3 - '1'
Traceback (most recent call last):
  File "<stdin>", line 1, in <module>
TypeError: unsupported operand type(s) for -: 'int' and 'str'
\end{minted}

Dynamic and static typing describe at what point the types need to be known.
With statically typed languages, types need to be specified at compile time
(such as \mintinline{c}{void func(int a)} in C). In more modern statically typed
languages, types can often be inferred automatically instead of having to be
specified by the programmer.

Python started as a dynamically typed language, which means that types are only
known at runtime. However, with Python 3.0 (released in 2008), syntax for
function annotations was added. Thus, the following code raised a
\verb|SyntaxError| in Python 2, but is valid syntax in Python 3:

\begin{minted}{python}
def add(a: int, b: int) -> int:
    return a + b
\end{minted}

However, how the type annotation syntax would be used was left to the
applications using them. Annotations are completely ignored at runtime, which
means a call like \py{sub("1", "2")} would not be rejected based on annotations
(and return the string \py{"12"}).

The annotation feature was not commonly used, but it started to gain traction
in 2014 with PEP 484\footnote{\url{https://www.python.org/dev/peps/pep-0484/}}.
``Python Enhancement Proposals'' (PEPs) are commonly written as a discussion
ground for larger changes to the Python language or ecosystem.

PEP 484 was accepted in May 2015, and a \verb|typing| module used to specify types
was added to the Python 3.5 standard library, released in September 2015. The
introduction of the \verb|typing| module allowed to specify types such as
\py{Optional[int]} (either an integer, or the special \py{None} value) or
\py{Mapping[str, int]} (a mapping from strings to integers).

However, Python remains a dynamically typed language, and those annotations are
still ignored at runtime. Apart from serving as documentation, there are various
projects interpreting them:

\label{typecheck-tools}
\begin{itemize}
  \item The PyCharm\footnote{\url{https://www.jetbrains.com/pycharm/}} IDE by
    JetBrains uses them for autocompletion and to notify the user about type
    errors.
  \item mypy\footnote{\url{http://www.mypy-lang.org/}} is the de-facto standard
    tool used to check type annotations for correctness: It can be run
    independently from Python and fails if there is a mismatch between the
    annotations and the implementation. Much of its development is supported by
    Dropbox.
  \item pytype\footnote{\url{https://github.com/google/pytype}} is an
    alternative to mypy, written by Google.
  \item pyre\footnote{\url{https://pyre-check.org/}} is another alternative
    to mypy, written by Facebook.
\end{itemize}

Python's approach of not requiring type annotations, as well as delegating their
interpretation to an external tool, is commonly referred to as
\emph{gradual typing}. A similar approach can be seen in Microsoft's
TypeScript\footnote{\url{http://www.typescriptlang.org/}} language, which is
type-checked and then compiled to JavaScript.

With both mypy and typescript, all type annotations are optional. Types which
are not annotated are assumed to be of a special type \verb|Any|. This type is
compatible with any other type (thus, an object of the type \verb|Any| can for
example passed as an argument annotated with \verb|int|), and also supports all
operations. In other words, objects of type \verb|Any| are not type-checked at
all.

Additionally, by default mypy only type-checks the inside of functions which
themselves have type annotations, in order to avoid flooding its user with
false-positives.

This approach allows to incrementally introduce ``static'' typing into an
existing codebase which previously had no type annotations at all, gradually
introducing more type annotations over a longer time, fixing errors as they
appear.

Since type checking in Python is a relatively new feature, not all third-party
libraries used by qutebrowser come with type information. For Python's standard
library and some third-party libraries, the
typeshed\footnote{\url{https://github.com/python/typeshed}} project exists,
where type information is maintained separately from Python itself.

\label{pep561}
Additionally, a more recent change to the Python ecosystem (PEP
561\footnote{\url{https://www.python.org/dev/peps/pep-0561/}}, accepted in June
2018) also allows type information (so-called ``stub files'' with a \verb|.pyi|
extension, similar to \verb|.h| header files in C or C++) to be distributed
independently from a library package itself.

\subsection{Introducing type checking}
As per the task description, type annotations were introduced into qutebrowser's
code base. Before starting this project, very few functions already carried type
annotations (since they were contributed by developers who prefer using them),
but the majority of qutebrowser's code was unannotated. Additionally, no type
checker was run systematically as part of qutebrowser's continuous integration
toolchain.

The tools mentioned in \ref{typecheck-tools} were evaluated, with the choice
falling on mypy since it is widely used and backed by a more diverse community
than either pytype and pyre (which are mostly developed and used by Google and
Facebook, respectively).

The introduction of mypy was split into several phases, with the goal of getting
mypy to run without any errors/warnings as quickly as possible, and then making
its checking stricter or more accurate.

\subsubsection{Initial run}
When running mypy over qutebrowser's codebase without any changes, 324 errors
were reported. Most errors looked similar to:

\begin{quote}
\begin{verbatim}
qutebrowser/utils/qtutils.py:36: error:
No library stub file for module 'PyQt5.QtCore'
\end{verbatim}
\end{quote}

with mypy complaining that it found no type information at all for various PyQt
modules.

\subsubsection{Ignoring missing imports}

Mypy provides a \verb|--ignore-missing-imports| switch which allows ignoring
such missing modules. Note that its usage is discouraged in mypy's documentation:
\emph{``We recommend using this approach only as a last resort''}\footnote{\url{https://mypy.readthedocs.io/en/latest/running_mypy.html\#missing-imports},
accessed 2018-12-03}. Nevertheless, it was deemed useful in order to focus on
the other existing error messages.

With the \verb|--ignore-missing-imports| option, mypy only produced 88 errors.

Analyzing those errors revealed that they could be split into various
categories:

\begin{itemize}
  \item Mypy not recognizing how qutebrowser adds an additional \verb|VDEBUG|
    (verbose debugging) logging level to Python's logging system. This was
    solved by adding \py{# type: ignore} comments to make mypy ignore those
    issues as appropriate.
  \item Trouble with a common pattern of handling optional library imports like so:
    \begin{minted}{python}
try:
    import secrets
except ImportError:
    secrets = None
\end{minted}
    There, mypy complained since setting a module type to the
    \verb|None| value is an invalid operation from the type system's view, but
    it still perfectly valid (if later guarded via \py{if secrets is not None:})
    in Python. Thus, those issues also were ignored via \py{# type: ignore}
    comments. An issue for better handling of this case in mypy was already
    open in mypy's issue tracker: \url{https://github.com/python/mypy/issues/1153}
  \item Mypy not handling Python's \py{enum.IntEnum} correctly when used via
    its function-based (rather than class-based) API. This was worked around by
    using the class-based API instead; an issue was already open in mypy's issue
    tracker: \url{https://github.com/python/mypy/issues/4865}.
  \item Additional type annotations being needed by mypy in places where types
    could not be inferred (such as \mintinline{python}{cmd_dict = {} # type: typing.Dict[str, Command]}).
  \item Module-level globals which are set to \py{None} initially and then
    initialized in a \py{init()} method. Mypy asserted that those could be \py{None}
    throughout the codebase (thus resulting in many potentially invalid operations),
    while their initialization happens very early, so they can be safely assumed
    to never be \py{None} despite their initial value. This was solved by
    overriding the assumed type via \py{instance = typing.cast('Config', None)}.
  \item Other values which similarly never are \py{None} when a certain function
    gets called, despite them having been \py{None} at some point before. Since
    mypy understands additional hints in the form of \py{assert val is not
      None} or \py{assert isinstance(index, int)}, such assertions could be used
    to silence those errors.
  \item One case were an existing annotation falsely claimed that a value could
    be either \py{str} or \py{int} (\py{typing.Union[str, int]}), while it
    always was a string.
  \item Various other cases where existing type annotations had to be refined to
    be more accurate.
\end{itemize}

After fixing those issues, mypy finished without showing any errors, but without
having type information for libraries being used by qutebrowser.

\subsubsection{Adding library type information}

After removing the \verb|--ignore-missing-imports| switch again, the main issue
was missing stubs for PyQt, the main library being used by qutebrowser. While
PyQt comes with a way to generate stub files from C++ type information (being a
quite simple wrapper over C++ code), those stubs come with various issues making
them unsuitable for use with mypy.

As a substitute, a PyQt5-stubs project
exists\footnote{\url{https://github.com/stlehmann/PyQt5-stubs/}}, packaging
adjusted type stubs for PyQt5 as a separate package (via PEP 561, see
\ref{pep561}).

With those installed and the switch removed, 55 new errors appeared. Those again
could be classified into various categories:

\begin{itemize}
  \item Issues with the PyQt5-stubs project where type annotations were
    inaccurate. Those were fixed in a copy (GitHub fork) of the stubs in the
    qutebrowser
    organization\footnote{\url{https://github.com/qutebrowser/PyQt5-stubs}} and
    contributed to the upstream project. However, a reply from its maintainer is
    still pending as of December 2018, so qutebrowser currently installs its own
    copy instead.
  \item Missing stub files for Python standard library modules in the typeshed
    project. For the \verb|faulthandler| module, writing stubs was
    straightforward, so those were
    contributed\footnote{\url{https://github.com/python/typeshed/pull/2627}} to
    the typeshed project. For more complex cases, those modules were ignored in
    the mypy config file instead.
  \item Bugs in the type annotations in typeshed, for which fixes were
    contributed\footnote{\url{https://github.com/python/typeshed/pull/2635}, \\
      \url{https://github.com/python/typeshed/pull/2636}} as well.
  \item Missing type annotations for other third-party modules. For those, an
    issue was opened in order to inform the respective developers about the
    possibilities to add type information to their packages, and the module was
    then ignored via the mypy config file.
  \item New places where mypy needed some additional type annotations in
    qutebrowser's code base for types it could not infer correctly.
\end{itemize}

\subsection{Increasing strictness}

After returning to a clean mypy invocation with additional type information, it
was attempted to add the \verb|--strict| switch to mypy, causing it to be more
pedantic about various checks. However, after enabling the strict option, mypy
displayed 5385 new errors. Most new errors were due to untyped functions (and
calls into such functions) being prohibited entirely in that case.

Instead, the more fine-grained options implied by \verb|--strict| were evaluated
individually:

\begin{description}
  \item[--warn-unused-configs] warns about unused module-specific settings in
    the config file, which could be caused by a typo. However, this could not be
    turned on due to a
    bug\footnote{\url{https://github.com/python/mypy/issues/5957}} causing
    false-positive warnings in mypy.
  \item[--disallow-subclassing-any] disallows subclassing from an object of an
    unknown (\verb|Any|) type. This was enabled globally except for a
    module (\verb|browser.webkit.rfc6266|) in qutebrowser describing a parser
    grammar, which relies on dynamic typing.
  \item[--disallow-untyped-calls] disallows calls into any function which does
    not have type information. This was causing a huge number of errors (since
    most of qutebrowser's codebase is not annotated yet), and thus was not
    enabled.
  \item[--disallow-untyped-defs] disallows defining any function without type
    annotation. This was not enabled since qutebrowser's codebase will gradually
    become more annotated. However, it was enabled for modules which were
    annotated as a part of this SA, so that any future additions to those
    modules will also need to carry type information.
  \item[--disallow-incomplete-defs] disallows partially annotated functions,
    where only some arguments are annotated. This was initially enabled and
    some partial annotations were completed, but it was later disabled due to
    false-positives in
    mypy\footnote{\url{https://github.com/python/mypy/issues/5954}}.
  \item[--check-untyped-defs] causes mypy to check the inside of functions which
    are not annotated yet. This caused 392 new errors and thus could not be
    enabled yet. However, reviewing those errors revealed that many were
    false-positives or mypy limitations, but also uncovered a crash (which was
    then fixed) in qutebrowser.
  \item[--disallow-untyped-decorators] disallows applying a decorator to a
    function which strips its type information (since the type of the decorator
    is unknown). This was enabled as it did not lead to any new errors.
  \item[--no-implicit-optional] disallows annotations with a \py{None} default
    value. An example of such an annotation is \py{def f(a: str = None)}.
    Instead, this option requires the more accurate \py{typing.Optional[str]}
    annotation. Since this would lead to more verbose type annotations without any
    perceived benefit (the same information can be seen by looking at the default
    value), it was not enabled.
  \item[--warn-redundant-casts] warns about constructs like \py{typing.cast(int,
      x)} when \py{x} already is of type \py{int}. It was enabled since it
    did not result in any new errors.
  \item[--warn-unused-ignores] warns about \mintinline{python}{# type: ignore}
    comment on lines where no error would have had occurred. This was enabled as
    it did not result in any new errors.
  \item[--warn-return-any] warns when a function returns a value of unknown
    (\verb|Any|) type, since that can propagate into other places which then are
    not type-checked either. This was not enabled as doing so would require many
    additional type annotation for qutebrowser's codebase.
\end{description}

\subsection{Adding annotations}

After mypy's configuration was completed and it still ran without any issues,
additional type annotations were introduced in qutebrowser's codebase. The task
description requires anything exposed by the extension API to carry type
annotations. Additionally, much of the config system (which is not exposed yet)
was also annotated in order to gain some additional experience with mypy.

This resulted in the following modules being fully annotated (in addition to
some being partially annotated):

\begin{itemize}[parsep=5pt]
  \item \texttt{browser.browsertab} (``tab API'')
  \item \texttt{browser.webelem}, \texttt{browser.webkit.webkitelem},
    \texttt{browser.webengine.webenginelem} (``web element API'')
  \item \texttt{misc.objects} and \texttt{commands.cmdutils} (registering commands)
  \item All modules in the \texttt{config} package: \\ \texttt{config}, \texttt{configcache},
    \texttt{configcommands}, \texttt{configdata}, \texttt{configdiff},
    \texttt{configexc}, \\ \texttt{configfiles}, \texttt{configinit}, \texttt{configtypes},
    \texttt{configutils}, \texttt{websettings}.
  \item All modules in the \texttt{api}, \texttt{components}
    and \texttt{extensions} sub-packages (see section \ref{sec:important})
\end{itemize}


% Implementation: Erläuterungen wichtiger konkreter Klassen
\section{Important packages, modules and classes}
\label{sec:important}
The modules added for qutebrowser's extension API are spread across three
packages:

\begin{itemize}
\item The \verb|qutebrowser.api| package contains the API exposed to
  extensions (in other words, any code which is part of the extension API).
\item The \verb|qutebrowser.extensions| package contains supporting infrastructure and
  internal code for handling extensions.
\item The \verb|qutebrowser.components| package contains modules which formerly
was part of qutebrowser's core and only uses the extension API -- that is, it
only imports code from the \verb|qutebrowser.api| package, but not from any
other \verb|qutebrowser.*| packages.
\end{itemize}

\subsection[The qutebrowser.api package]{The qutebrowser.api package: Public extension API}

\begin{table}[H]
  \centering
  \begin{tabulary}{\linewidth}{lL}
    \toprule
    Module & Description \\
    \midrule
    \verb|apitypes.py| & Various basic types which can be used by extensions.
                         Those are either used as type annotations (such as
                         \verb|Tab|, which is an \verb|AbstractTab| from figure
                         \ref{fig:tabapi-inherit}) or as enumerations (such as
                         \verb|ClickTarget| which is used to specify how to open
                         a clicked link). \\
    \verb|cmdutils.py| & Utilities related to registering command handlers from
                         extensions, such as the \py{@cmdutils.register()}
                         decorator. \\
    \verb|config.py| & Access to the config from extensions, by either using a
                       shorthand like
                       \verb|config.val.content.javascript.enabled|, or the
                       \verb|get()| function like
                       \py{config.get('content.javascript.enabled')}. \\
    \verb|downloads.py| & Used to trigger download of temporary files (such as
                          adblock filter lists) from extensions. In the future,
                          this module could be extended to allow interacting
                          with existing downloads triggered by the user. \\
    \verb|hook.py| & Allows extensions to register hooks for certain events via
                     decorators, such as \py{@hook.init()} or
                     \py{@hook.config_changed()} \\
    \verb|interceptor.py| & Can by used by extensions to register a
                            \emph{request interceptor}, which then gets called
                            for every network request made by qutebrowser. Based
                            on the URL of the page and the URL being requested,
                            the extension may decide to block the request. \\
    \verb|message.py| & Used to show messages to the user via functions like
                        \py{message.info("...")} or \py{message.error("...")}. \\
    \bottomrule
  \end{tabulary}
  \caption{Modules in the qutebrowser.api package.}
  \label{tab:apimodule}
\end{table}

The \verb|qutebrowser.api| package is described in more detail in the developer
documentation in appendix \ref{ch:sphinx}.

\subsection[The qutebrowser.extensions package]{The qutebrowser.extensions package: Internal extension machinery}

The \verb|qutebrowser.extensions| package consists of two modules, which are
explained in further detail below.

\subsubsection{extensions.interceptors module}
This module implements the internal logic so extensions can intercept and
optionally block network requests made by qutebrowser.

\begin{table}[H]
  \centering
  \begin{tabulary}{\linewidth}{lL}
    \toprule
    Class / Function & Description \\
    \midrule
    \verb|Request| & A class representing a network request, containing
                     information such as \verb|first_party_url| (the page being
                     visited) or \verb|request_url| (the URL of the resource
                     being requested). The request can be blocked by calling its
                     \verb|block()| method. \\
    \verb|InterceptorType| & The type of an interceptor function, intended to be
                             used in type annotations (exposed to extensions as
                             \verb|qutebrowser.api.interceptor.InterceptorType|). \\
    \verb|register()| & Register a new request interceptor (exposed to
                       extensions as \verb|qutebrowser.api.interceptor.register()|) \\
    \verb|run()| & Used internally by qutebrowser to run all registered
                  interceptors over a request. \\
    \bottomrule
  \end{tabulary}
  \caption{Classes and functions in the qutebrowser.extensions.interceptors package.}
\end{table}

\subsubsection{extensions.loader module}
This module implements the internal loading and initializing of
extensions/components. It is responsible for dynamically finding all modules
in the \verb|qutebrowser.components| package, loading them, and calling their
registered hooks correctly.

\begin{table}[H]
  \centering
  \begin{tabulary}{\linewidth}{lL}
    \toprule
    Class / Function & Description \\
    \midrule
    \verb|InitContext| & Information passed to an extension when it gets
                         initialized (if it declares a function decorated with
                         \py{@hook.init()}, see \verb|hook.py| in table
                         \ref{tab:apimodule}). Contains information such as the
                         commandline arguments passed to qutebrowser, or the
                         data/config directories used. Used via a
                         \verb|_get_init_context()| factory method.\\
    \verb|ModuleInfo| & Information attached to a Python module object. It is
                        used to record internal information by decorators like
                        \py{@hook.init()} (see \verb|hook.py| in table
                        \ref{tab:apimodule}). \\
    \verb|ExtensionInfo| & Meta-information about an extension. Currently only
                           contains the name of an extension, but could be
                           extended in the future to record additional
                           information such as version numbers or the author of
                           a third-party extension. \\
    \verb|add_module_info()| & Used internally to add a \verb|ModuleInfo|
                               instance to a Python module object. \\
    \verb|load_components()| & Finds and loads all modules in the
                               \verb|qutebrowser.components| package (see
                               section \ref{sec:components}). Uses a
                               \verb|_load_component()| utility function
                               internally which loads a single component. \\
    \verb|walk_components()| & Find all available components. Used by
                               \verb|load_components()| and by qutebrowser's
                               packaging infrastructure so all component modules
                               are added to Windows/macOS builds (via
                               PyInstaller). Uses two different implementations
                               internally: \verb|_walk_normal()| and
                               \verb|_walk_pyinstaller()|. The latter needs to
                               be used because the simpler approach doesn't work
                               in builds built via
                               PyInstaller
                               \footnote{\url{https://github.com/pyinstaller/pyinstaller/issues/1905}}. \\
    \verb|_on_config_changed()| & Triggered on a configuration change, takes
                                  care of finding and calling all extension
                                  methods decorated with
                                  \py{@hook.config_changed()} (see
                                  \verb|hook.py| in table \ref{tab:apimodule}). \\
    \bottomrule
  \end{tabulary}
  \caption[Important classes and functions in the qutebrowser.extensions.loader
  package.]{Important classes and functions in the qutebrowser.extensions.loader package.
    Some private functions were redacted for brevity.}
\end{table}

\subsection[The qutebrowser.components package]{The qutebrowser.components
  package: Code moved out of the core}
\label{sec:components}

\begin{table}[H]
  \centering
  \begin{tabulary}{\linewidth}{lL}
    \toprule
    Module & Description \\
    \midrule
    \verb|adblock.py| & qutebrowser's adblocker implementation, blocking
                         advertisements in websites. \\
    \verb|caretcommands.py| & Commands related to moving the caret
                              (cursor) around via keybindings, such as
                              \verb|:move-to-end-of-document| or
                              \verb|:toggle-selection| (18 commands total). \\
    \verb|misccommands.py| & Miscellaneous qutebrowser commands, such as
                             \verb|:home|, \verb|:reload| or \verb|:print| (15
                             commands total) \\
    \verb|scrollcommands.py| & Commands related to scrolling (\verb|:scroll|,
                               \verb|:scroll-px|, \verb|:scroll-to-perc|,
                               \verb|:scroll-to-anchor|) \\
    \verb|zoomcommands.py| & Commands related to zooming (\verb|:zoom-in|,
                             \verb|:zoom-out|, \verb|:zoom|) \\
    \bottomrule
  \end{tabulary}
  \caption{Modules in the qutebrowser.components package.}
\end{table}

% Automatische Testverfahren
\section{Automated Testing}
When this project was started, the qutebrowser project already existed -- it was
started in December~2013. Thus, a comprehensive test suite was already present
at the time:

\begin{itemize}[parsep=5pt]
  \item Around 7400 tests (including parametrizations with different data)
  \item Including fuzzing (via hypothesis\footnote{https://hypothesis.works/})
    and benchmark tests (via pytest-benchmark\footnote{https://pytest-benchmark.readthedocs.io/})
  \item Testing on Linux, macOS and Windows
  \item Testing with all supported Python versions (3.5, 3.6 and 3.7)
  \item Testing with all supported Qt versions (5.7, 5.9, 5.10, 5.11, 5.12)
  \item Average test coverage of 80\% (including branch coverage)
\end{itemize}

Therefore, the aim in this project was to build upon the existing testing
infrastructure, and make sure the total coverage does not decrease with the
newly added code. Ideally, the coverage of newly added modules should surpass
the existing average coverage of 80\%\footnote{The author is well aware that
  coverage alone is not necessarily a useful metric. Blindly following a
  ``coverage goal'' thus is of little value. However, it is useful when using it
  as a guideline while consciously developing good tests, which is what has been
  done.}.

Unfortunately, there were no good uses for the hypothesis (fuzzing, i.e.,
testing with many random values) and pytest-benchmark libraries while adding
tests for the code added/changed during this project.

The test coverage for modules added for the extension API (or modules with major
changes) is shown in table \ref{tab:coverage}.

% >>> v = [100, 100, 100, 70, 90, 100, 100, 93, 82, 73, 96, 85, 94, 88]
% >>> sum(v)/len(v)
% 90.78571428571429
\begingroup
\renewcommand{\arraystretch}{1}
\begin{table}[H]
  \centering
  \begin{tabulary}{\linewidth}{rlrL}
    \toprule
    Directory & File & Coverage (\%) & Comment \\
    \midrule
    api/ & \emph{Average} & 80 & \\
    & apitypes.py & 100 & \\
    & cmdutils.py & 100 & \\
    & config.py & 100 & \\
    & downloads.py & 70 & Requires a download manager object for which no
                          stub/mock exists yet. \\
    & hook.py & 90 & \\
    & interceptor.py & 100 & \\
    & downloads.py & 100 & \strut\vspace{1em} \\
    components/ & \emph{Average} & 85 & \\
    & adblock.py & 93 & \\
    & caretcommands.py & 82 & \\
    & misccommands.py & 73 & Some commands (like printing) contain
                             hard to test GUI-interactions \\
    & scrollcommands.py & 96 & \\
    & zoomcommands.py & 85 & \strut\vspace{1em} \\
    extensions/ & \emph{Average} & 89 & \\
    & interceptors.py & 94 & \\
    & loader.py & 88 & \strut\vspace{1em} \\
    browser/ & browsertab.py & 87 & \\
    commands/ & command.py & 89 & \\
    \midrule
    Average & & 90 & \\
    \bottomrule
  \end{tabulary}
  \caption{Test coverage for added/changed modules}
  \label{tab:coverage}
\end{table}
\endgroup

Based on this data, the goal of surpassing the existing average coverage of 80\%
was met: In the files relevant to this SA, an average test coverage of 90\% has
been achieved.

A test output log can be found in appendix \ref{ch:testlog}.

% Manuelle Testverfahren, etc.
\section{Manual Testing}
The following tests were (successfully) performed manually, since related
automated tests were insufficient:

\begin{itemize}[parsep=5pt]
  \item Updating adblock filter lists via the \verb|:adblock-update| command;
    verifying that lists are downloaded correctly and a \texttt{adblock: Read ...
    hosts from 1 sources.} message is shown.
  \item Verifying that visiting a website containing ads
    (\url{https://www.20min.ch/}) is displayed without ads.
  \item Running \verb|:print| to display the print dialog and printing the
    website into a PDF file; using the \verb|:print --pdf| file to generate a
    PDF file directly.
\end{itemize}
