% -*- TeX-master: "main.tex" -*-

% Bewertung (Evaluation)
\chapter{Evaluation}
\label{ch:evaluation}

% Kriterien (Wie wird bewertet?)
\section{Criteria}

\label{criteria}

There are various forces which affect design decisions for qutebrowser's
extension API:

\begin{description}
\item[Qt APIs] qutebrowser is built on top of the QtWebEngine/QtWebKit
rendering engines (which of the two to use is user-configurable). Sometimes,
while a clean API design for a given problem would exist, constraints imposed by
those libraries are a limiting factor and thus influence the API design.
The API to get the selected text from a web page is an ideal example: The most
straightforward API would be a \py{def selection() -> str} method. However,
JavaScript execution is needed to get the selection, which is only available
from Qt as a callback-based interface. Thus, the extension API will need to look
like \py{def selection(cb: Callable[[str], None])} \\ \py{-> None} -- in other words,
the \py{selection} method will take a callback function, which then gets called
with the selected text.

\item[Internal qutebrowser code] One of the main goals (as per the task
description) of this SA is moving code from qutebrowser's core into internal
``extensions'' shipped alongside qutebrowser. Components which use general-purpose
APIs (like the adblocker, which needs to intercept network requests) can
conveniently moved out of the core, and result in extension APIs which are also
usable for other purposes.

\item[Ideas for future extensions] While external extensions (contributed by the
qutebrowser community) are not the primary focus of this SA, a lot of use-cases
for extensions have been collected based on users' feature requests. Care should be
taken so the extension API can also satisfy these use cases in the future.

\item[Other extension APIs] There is a general consensus from browser vendors
around the WebExtension API. Unfortunately, that API is unsuitable for
qutebrowser, for reasons explained in chapter \ref{unsuitable}. It can still
serve as a source for inspiration.

\item[Python and Qt] While some higher-level architectural decisions are
independent from the programming language used to implement them, what is
commonly considered a ``good'' API certainly depends on the underlying
programming language and the idioms used therein. Since qutebrowser's extension
API is used from Python, it should aim to be ``Pythonic'' (i.e., adhering to
Python idioms\footnote{\url{https://blog.startifact.com/posts/older/what-is-pythonic.html}})
and also use features made available by Qt. As an example, a Pythonic API might
favor Python
decorators\footnote{\url{https://docs.python.org/3/glossary.html\#term-decorator}}
over inheritance to set up extension hooks; or a Qt API might prefer Qt's
signals/slots
facility\footnote{\url{https://doc.qt.io/qt-5/signalsandslots.html}} over a
callback-based API.
\end{description}

In addition to the mentioned external forces, it's helpful to have a set of
self-imposed design guidelines in mind when designing an API. For this project,
it was decided to adopt the \emph{``Characteristics of good APIs''} listed in
Jasmin~Blanchette's \emph{Little Manual of API Design}
\autocite[7ff]{api-design}:

\begin{description}
  \item[Easy to learn and memorize:] This also implies consistency and minimalism,
    as well as clear semantics, following the principle of least surprise.
  \item[Leads to readable code:] As the manual puts it: \emph{``Readable code can be
  concise or verbose. Either way, it is always at the right level of abstraction
  -- neither hiding important things nor forcing the programmer to specify
  irrelevant information''}.
  \item[Hard to misuse:] A well-designed API should assist its user in
    writing correct and clear code. 
  \item[Easy to extend:] New concepts will appear over time, and existing
    APIs will grow. While it's hard to anticipate the future, future
    extendability should be taken into account while designing an API.
  \item[Complete:] The manual claims \emph{``Ideally, an API should be
    complete and let users do everything they want.''}. However, this statement
    is later revised by adding \emph{``Completeness is also something that can appear
    over time, by incrementally adding functionality to existing APIs. However,
    it usually helps even in those cases to have a clear idea of the direction
    for future developments, so that each addition is a step in the right
    direction.''}
\end{description}

% Schlussfolgerungen, eigener Lösungsansatz
\section{Conclusion}
Designing an API always is a trade-off between various, sometimes conflicting,
forces. For qutebrowser's extension API, the focus should lie on being as
minimal and straightforward as possible, with the primary goal of moving
internal qutebrowser code to extensions.

While extensibility and completeness are valid concerns, they are less relevant
as the API is not open to third-party extensions yet.
