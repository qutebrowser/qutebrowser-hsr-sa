% -*- TeX-master: "main.tex" -*-

\chapter{Requirements Specification}
\label{ch:requirements}

% Use Cases (Success Scenario / Success Diagram)
\section{Use Cases}

This project extends an existing codebase with an extension API rather
than starting a new project from scratch. By its nature, it's difficult to
predict how an extension API will be used in the future. Because of that, an
use-case diagram would not adequately describe the motivation for the extension
API.

Various ideas for future third-party extensions have been voiced by the
qutebrowser community; they are collected in a GitHub
issue\footnote{\url{https://github.com/qutebrowser/qutebrowser/issues/30},
  accessed 2018-11-08}. However, the main aim of this project (and thus the main
focus for the extension API) is reducing the complexity of qutebrowser's core.

% Weitere Funktionen, die nicht erfasst wurden
%\section{Further Functionality}

% Nicht-funktionale Anforderungen (Rahmbenbedingungen, evtl. Verweis auf 1.3)
\section{Non-functional Requirements}
The following non-functional requirements are relevant for qutebrowser's
extension API:

\begin{description}
  \item[Security] The security model used for qutebrowser extensions assumes
    that extensions are trusted, i.e., may run arbitrary code. See
    section \ref{security} for a more detailed explanation.
  \item[Simplicity] It should be trivial for a user to extend her qutebrowser
    setup with a custom extension. Thus, getting started with writing a
    third-party extension should be as straightforward as possible, without
    requiring packing multiple files into a custom format. Also see section
    \ref{anatomy} for an explanation on how this topic is handled for
    WebExtensions, and how qutebrowser's API differs from that.
  \item[Learnability] When the extension API is opened for third-party
    contributions, it should be easy to get accustomed to it. Therefore, the API
    should be intuitive for Python programmers, and well documented.
  \item[Compability] The qutebrowser project runs on a variety of different
    software versions -- various operating systems (Linux, macOS, Windows,
    and more) are supported, including different Qt and Python versions. The
    extension API should abstract over these differences, so an extension
    written for qutebrowser (or core code moved into such an extension) runs in
    all situations qutebrowser itself can run.
  \item[Performance] Having code in extensions (rather than in the core part of a
    project) can result in performance degradations due to more moving parts and
    more layers being involved. Since those impacts are usually small
    individually (but might be big enough to be relevant collectively), no
    relevant impact is expected in the scope of this SA. However, performance
    considerations might be a good reason to keep some code in the core rather
    than moving it to extensions in the future.
\end{description}

% % Detailspezifikation
% \section{Detailed Specification}
% \fixme{???}
% 
% % Analyse (Business Model)
% \chapter{Analysis}
% 
% % Domain Modell, Klassendiagramme (konzeptionell)
% \section{Domain Model}
% 
% % Objektkatalog (Beschreibung der Konzepte, bzw. Entitätsmengen)
% \section{Objects}
