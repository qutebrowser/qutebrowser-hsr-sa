% -*- TeX-master: "main.tex" -*-

% Einführung
\chapter{Introduction}
\label{ch:intro}

\section{Background}

The qutebrowser project is a web browser, comparable to Google Chrome or Mozilla
Firefox, which is focused on being keyboard-driven and having a minimal user
interface. It is aimed at power users who value customizability and efficiency,
but are willing to accept a rather steep learning curve compared to
``traditional'' web browsers.

Its philosophy and keybindings are inspired by the Vim
editor\footnote{\url{https://www.vim.org/}} which is available on almost any
UNIX system (such as macOS or Linux). Similar projects exist as addons for
Firefox (such as Tridactyl\footnote{\url{https://github.com/tridactyl/tridactyl}})
or Chrome (Vimium\footnote{\url{https://vimium.github.io/}}). However, due to
being constrained by the extension API exposed by Firefox and Chrome, those
projects lack various features which qutebrowser offers.

Since qutebrowser uses the Python programming language in conjunction with the
Qt library for graphical user interfaces, it is standing on the shoulders of
giants: It does not implement complex tasks such as downloading and executing
HTML/CSS/JavaScript code itself. Instead, it relies on the QtWebEngine project
to do so, which is largely based on the same code as Google Chrome.

An extension API for users to write their own extensions to qutebrowser is a
long-standing feature
request\footnote{\url{https://github.com/qutebrowser/qutebrowser/issues/30}},
which has often been requested by its users.

The original inspiration for qutebrowser was the \emph{dwb}
project\footnote{\url{https://portix.bitbucket.io/dwb/}} which is a web browser
with a very similar philosophy. The author of qutebrowser used dwb for a long
time, but it became apparent that dwb's author lacked the time and interest to
continue working on it, thus many dwb users were looking for a viable
alternative. Since no suitable alternative existed, work on qutebrowser started
in December 2013, after which it quickly gained traction.

It is difficult to estimate qutebrowser's user count, but it is most likely used by a
couple thousand users, so an extension API is also vital in order to be able to move
less popular features out of the core part of qutebrowser.

\label{backends}
There are two backends (rendering engines) which can be used with qutebrowser:

\begin{itemize}
  \item \emph{QtWebKit} which is based on the
  WebKit\footnote{\url{https://www.webkit.org/}} project
  \item \emph{QtWebEngine} which is based on the
  Chromium\footnote{\url{https://www.chromium.org/}} project, which is also used
  in Google Chrome (used as default).
\end{itemize}

In order to allow using either backend, qutebrowser provides an abstraction
layer over the two libraries, implementing the adapter pattern
\autocite[139ff]{gof}. This abstraction layer is referred to as ``tab API'', and
documented in section \ref{tabapi}.

% Problemstellung, Vision
\section{Vision}
\label{vision}

Many qutebrowser users are power-users and, as such, have very specific (and
sometimes unique) feature requests and workflows. It should be possible for
those users to extend qutebrowser with custom extensions in an easy way, in order
to keep qutebrowser's core small.

Since qutebrowser already has a thriving community, this change also intends to
decentralize development efforts, as it enables users and developers to maintain
their extensions independently from the core development.

% Ziele und Unterziele
\section{Objectives}
\label{goals}

The official task description (``Aufgabenstellung'') in German can be found in
appendix \ref{ch:aufgabenstellung}.

\subsection{Merging contributions}
When this project started, various third-party contributions were pending a code
review, due to being ignored during the exam session preceding this semester.

Since those contributions would conflict with the refactorings needed before
work on the extension API can start, those should be reviewed and merged (or
rejected) at the beginning of this project.

\subsection{Refactorings}
\label{sec:goals-refactorings}

Initially, qutebrowser was developed without knowledge of proper software
engineering practices, which resulted in some maintainability issues. While many
of those issues have since been resolved, some still remain. The refactorings
needed to fix those issues will affect the API exposed to extensions, and
therefore should be taken care of before attempting to design a extension API.

The following changes are planned:

\begin{itemize}
  \item \url{https://github.com/qutebrowser/qutebrowser/issues/1456}: \\
    Parts of qutebrowser already use Python type
    annotations\footnote{\url{https://www.python.org/dev/peps/pep-0484/}}, but
    only if contributors decide to use them. In addition to that, no type
    checker such as mypy\footnote{\url{http://mypy-lang.org/}} is currently run
    as part of qutebrowser's continuous integration (CI) pipeline, thus allowing
    regressions to occur. As part of this project, a type checker should be
    introduced into the CI infrastructure, and any code exposed via the extension
    API should be annotated with proper type annotations.
  \item \url{https://github.com/qutebrowser/qutebrowser/issues/345}: \\
    To generate HTML documentation, qutebrowser currently uses
    asciidoc\footnote{\url{http://asciidoc.org/}} which is unsuitable for API
    documentation and ceased maintenance. The
    Sphinx\footnote{\url{http://www.sphinx-doc.org/}} tool should be
    introduced instead, which is tailored for documenting APIs.
  \item \url{https://github.com/qutebrowser/qutebrowser/issues/640}: \\
    Global objects are registered in a object registry based on a name as
    string (``stringly-typed''\footnote{\url{http://wiki.c2.com/?StringlyTyped}}).
    This historically caused various object-lifetime related issues, and also
breaks tooling such as the mypy type checker. All code using the object registry
should be refactored to use better alternatives such as constructor arguments
(dependency injection).
\end{itemize}

\subsection{Extension API}
After the needed refactorings are finished, an extension API for qutebrowser
should be designed and implemented. In order to design the API with real-world
usages in mind, the following steps should be taken:

\begin{itemize}
  \item Identify what functionality to expose over the extension API. This can
    be done by analyzing code in qutebrowser's core which is suitable for moving
    into extensions. Use cases for future third-party extensions should be
    considered as well.
  \item Research existing APIs for browser extensions and analyze prior
    solutions.
  \item Design and implement a suitable API based on the knowledge gained.
  \item Use the newly added API by moving code out of qutebrowser's core into
    extensions (which are shipped with qutebrowser).
  \item Review the finished API to make sure it is consistent and intuitive.
\end{itemize}

\subsection{Documentation}

As shown in GitHub's ``Open Source Survey'' \autocite{oss-survey}, good
documentation is highly valued in open source projects, but also is frequently
overlooked by project maintainers.

Therefore, good documentation is an explicit goal of this project: The resulting
extension API should be documented well by providing API documentation,
usage examples, and an example extension.

Additionally, this report is written in English, allowing a bigger audience to
read it. Thus, future extension authors or qutebrowser contributors can read
this documentation to get further background information about its extension
API.

% Rahmenbedingungen, Umfeld, Definitionen, Abgrenzungen
\section{Context}
\label{context}

The software and version constraints are mostly given by the existing project:

\begin{itemize}[parsep=5pt]
  \item Python\footnote{\url{https://www.python.org/}} 3 (3.5 or newer)
  \item Qt\footnote{\url{https://www.qt.io/}} 5 (5.7 or newer), used via PyQt5\footnote{\url{https://www.riverbankcomputing.com/software/pyqt/intro}}
  \item pytest\footnote{\url{https://pytest.org/}} as test framework
  \item Various code quality tools: pylint\footnote{\url{https://pylint.org/}},
    flake8\footnote{\url{http://flake8.pycqa.org/}} and others.
\end{itemize}

As qutebrowser is a pre-existing project with a vibrant community, external
contributions are expected to continue (despite an
announcement\footnote{\url{https://lists.schokokeks.org/pipermail/qutebrowser-announce/2018-October/000053.html}, accessed 2018-11-12}
asking people to refrain from making larger contributions). This can be challenging,
as it results in refactorings being carried out against a moving target. Because
of the nature of open-source contributions,
% http://www.catb.org/~esr/writings/cathedral-bazaar/cathedral-bazaar/
% https://www.fordfoundation.org/about/library/reports-and-studies/roads-and-bridges-the-unseen-labor-behind-our-digital-infrastructure/
it is hard to foresee or control which areas external contributors are changing.
At the beginning of the SA, some time was allocated for merging external
contributions (pull requests) which were already open. For the rest of the SA,
such contributions will be dealt with on a best effort basis, with the main
focus being this documentation and the work required for the extension API.

\subsection{Notable language features}
\label{sec:langfeatures}
This document assumes that the reader is familiar with a programming language
following the object oriented paradigm and concepts such as classes.

Some sections refer to concepts which are specific to Python or Qt and are
thus explained in more detail below.

\subsubsection{Python decorators}
Python allows to annotate a function or method with a ``decorator''. As an
example, this can be used to define properties:

\begin{minted}{python}
class Example:

    @property
    def prop(self):
        return 2 + 2

ex = Example()
print(ex.prop)  # 4
\end{minted}

The same feature can also be used to implement custom decorators. Functions are
first-level objects in Python, thus a decorator receives the function it
decorates as an argument and returns a function (which may be modified).

A more thorough explanation of decorators can be found here: \\
\url{https://realpython.com/primer-on-python-decorators/}

\subsubsection{Python modules and packages}
A Python module simply is a file with a \verb|.py| extension containing Python
code. A package is a folder containing modules and a special \verb|__init__.py|
file (often empty) as a package marker.

\subsubsection{Qt signals and slots}
Signals and slots are an important feature of the Qt library to allow
independent objects to communicate with each other.

They can be considered an abstraction of the observer pattern
\autocite[293ff]{gof}, similar to delegates and events in C\#.

A class can define a signal which it uses to notify other objects (e.g. about
changed state):

\begin{minted}{python}
class Example(QObject):

    text_changed = pyqtSignal(str)
\end{minted}

A slot is a function or method which can be connected to a signal:

\begin{minted}{python}
@pyqtSlot(str)
def on_text_changed(text: str) -> None:
    print(text)
\end{minted}

Signals provide a \verb|connect()| method to connect them to a slot:

\begin{minted}{python}
ex = Example()
ex.text_changed.connect(on_text_changed)
\end{minted}

as well as an \verb|emit()| method used to call all slots connected to it -- the
above \verb|Example| class could have a \verb|set_text| method like this:

\begin{minted}{python}
def set_text(self, text: str) -> None:
    # ...
    self.text_changed.emit(text)
\end{minted}

More information about Qt signals and slots can be found in its (C++)
documentation: \\
\url{https://doc.qt.io/qt-5/signalsandslots.html}

% Vorgehen, Aufbau der Arbeit
\section{Methods and Structure}
\subsection{Structure of this document}
This document is organized in roughly chronological order.

In this chapter, the background necessary to understand the
project and its goals is explained. Chapter \ref{ch:projectman} documents the
project methodology used and allocates the time available.
It also analyzes the risks involved. Chapter \ref{ch:requirements} specifies the
project requirements in more detail. In chapter \ref{ch:existing}, existing
similar work is examined. Chapter \ref{ch:evaluation} notes the criteria being
considered for the project. Based on these criteria, a concept was designed in chapter
\ref{ch:concept}. Finally, chapter \ref{ch:implementation} explains the actual
implementation, while chapter \ref{ch:results} analyzes the results.

In appendix \ref{ch:glossary}, a glossary of the terms and abbreviations used
can be found.

\subsection{Design decisions}
In order to take informed and sustainable design decisions, the
\emph{Y-Approach} \autocite{yapproach} was evaluated. It proposes the following
template for design decisions:

\begin{quote}
  \begin{itemize}[parsep=5pt]
    \item In the context of \emph{use case $uc$ and/or component $co$},
    \item \ldots facing \emph{non-functional concern $c$},
    \item \ldots we decided for \emph{option $o_1$}
    \item and neglected \emph{options $o_2$ to $o_n$},
    \item \ldots to achieve \emph{quality $q$},
    \item \ldots accepting downside \emph{consequence $c$}.
  \end{itemize}
\end{quote}

The template serves as a good reminder of the aspects to keep in mind
while taking a decision. However, it was found to not be very helpful for
documentation, as a different structure (such as a list of considerations, or
prose text) is often more readable. Thus, it was used as an inspiration only,
but the explanations in this document do not strictly follow its structure.

\fixme{we don't need high-level decisions}

\subsection{Tools used}
Since this project was realized by a single author (rather than in a group),
project management tooling was kept to a minimum. GitHub issues and its project
board features were used for project management, along with
Clockify\footnote{\url{https://www.clockify.me/}} for time tracking.

This document was typeset in \LaTeX{} using the \emph{Computer Modern Bright}
font and various additional packages. GNU Emacs was used as text editor.

\fixme{Nutzungsrechte}
