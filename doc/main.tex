\documentclass[a4paper,parskip=full]{scrreprt}

\usepackage[T1]{fontenc}
\usepackage{cmbright}
\usepackage[utf8]{inputenc}
\usepackage{hyperref}
\usepackage{booktabs}
\usepackage{graphicx}
\usepackage{natbib}

\begin{document}

%\title{qutebrowser made extendible \\ FIXME}
%\author{Florian Bruhin \\ \url{florian@qutebrowser.org}}
%\date{\today}
%\maketitle

\begin{titlepage}

\begin{flushleft}

% Upper part of the page
\noindent\begin{minipage}[t]{0.49\textwidth}
	\begin{flushleft}
		\vspace{3pt} %needed else aligned to bottom
		\includegraphics[height=0.12\textheight]{img/hsr.eps}
	\end{flushleft}
\end{minipage}
\hfill
\begin{minipage}[t]{0.49\textwidth}
	\begin{flushright}
		\vspace{0pt} %needed else aligned to bottom
		\includegraphics[height=0.15\textheight]{img/qutebrowser.png}
	\end{flushright}
\end{minipage}
\\[4cm]

{\huge \bfseries qutebrowser made extendible}\\[0.5cm]
%{\large \bfseries Student Research Project (Studienarbeit)}\\[2cm]
{\large \bfseries Term Project (Studienarbeit)}\\[2cm]

Departement of Computer Science\\
University of Applied Sciences Rapperswil\\
\url{https://www.hsr.ch/}\\[1cm]

% Author and advisor
Author: Florian Bruhin\\[0.3cm]
Advisor: Prof.~Stefan Keller\\[0.3cm]
External Co-Examiner: Claude Eisenhut, Eisenhut Informatik AG

% Bottom of the page
\vfill
Date: {\today}

\end{flushleft}

\end{titlepage}



\chapter*{Abstract}

\chapter*{Management Summary}


\tableofcontents
\listoffigures
\listoftables


% Aufgabenstellung
\chapter*{Task Definition}  

% Technischer Bericht
\part{Technical Report}

% Einführung
\chapter{Introduction}

\section{Background}
The qutebrowser project exists since December 2013. It is a keyboard-focused
browser with a minimal GUI, based on Python and PyQt5.

A plugin API for users to write their own extensions to qutebrowser is a
long-standing feature
request\footnote{\url{https://github.com/qutebrowser/qutebrowser/issues/30}},
which has often been requested by its users.

It's difficult to estimate qutebrowser's user count, but it's likely used by a
couple thousand users, so a plugin API is also vital in order to be able to move
less popular features out of the core part of qutebrowser.

% Problemstellung, Vision
\section{Vision}

Many qutebrowser users are power-users and, as such, have very specific (and
sometimes unique) feature requests and workflows. It should be made possible for
those users to extend qutebrowser with custom plugins in an easy way, in order
to keep qutebrowser's core small.

Since qutebrowser already has a thriving community, this change also intends to
decentralize development efforts, as it makes it possible for power-users and
developers to maintain their plugins independently from the core development.

% Ziele und Unterziele
\section{Goals}

Initially, qutebrowser was developed without knowledge of proper software
engineering practices, which resulted in some maintainability issues. While many
of those issues have been cleaned up since, some still remain. Those
refactorings affect the API exposed to plugins, and therefore should be taken
care of before attempting to design a plugin API.

The full list of relevant refactorings is tracked as a Kanban
board\footnote{\url{https://github.com/qutebrowser/qutebrowser/projects/3}} in
qutebrowser's GitHub repository. The biggest planned changes are the following:

\begin{itemize}
  \item \url{https://github.com/qutebrowser/qutebrowser/issues/1456}: \\ Parts of qutebrowser already use Python type
    annotations\footnote{\url{https://www.python.org/dev/peps/pep-0484/}}, but
    only if contributors decide to use them. In addition to that, no type
    checker such as mypy\footnote{\url{http://mypy-lang.org/}} is currently run
    as part of qutebrowser's continuous integration (CI) pipeline, thus allowing
    regressions to occur. As part of this project, a type checker should be
    introduced into the CI infrastructure, and any code exposed via the plugin
    API should be annotated with proper type annotations.
  \item \url{https://github.com/qutebrowser/qutebrowser/issues/345}: \\
    To generate HTML documentation, qutebrowser currently uses
    asciidoc\footnote{\url{http://asciidoc.org/}} which is unsuitable for API
    documentation and ceased maintenance. An external contributor (see page
    \pageref{fiete}) is currently working on migrating to the
    Sphinx\footnote{\url{http://www.sphinx-doc.org/}} toolchain, and should be
    supported with his work throughout the SA.
  \item \url{https://github.com/qutebrowser/qutebrowser/issues/640}: \\
    Global objects are registered in a object registry based on a name as
    string (``stringly-typed''\footnote{\url{http://wiki.c2.com/?StringlyTyped}}).
    This historically caused various object-lifetime related issues, and also
breaks tooling such as the mypy type checker. All code using the object registry
should be refactored to use better alternatives such as constructor arguments
(dependency injection).
\end{itemize}

% Rahmenbedingungen, Umfeld, Definitionen, Abgrenzungen
\section{Context}

The software and version constraints are mostly given by the existing project:

\begin{itemize}
  \item Python\footnote{\url{https://www.python.org/}} 3 (3.5 or newer)
  \item Qt\footnote{\url{https://www.qt.io/}} 5 (5.7 or newer), used via PyQt5\footnote{\url{https://www.riverbankcomputing.com/software/pyqt/intro}}
  \item pytest\footnote{\url{https://pytest.org/}} as test framework
  \item Various code quality tools: pylint\footnote{\url{https://pylint.org/}},
    flake8\footnote{\url{http://flake8.pycqa.org/}} and others.
\end{itemize}

As qutebrowser is a pre-existing project with a vibrant community, community
contributions are expected to continue (despite an
announcement\footnote{\url{https://lists.schokokeks.org/pipermail/qutebrowser-announce/2018-October/000053.html}}
asking people to hold back bigger contributions). This can be challenging,
as it results in refactorings being carried out against a moving target. Because
of the nature of open-source contributions,
% http://www.catb.org/~esr/writings/cathedral-bazaar/cathedral-bazaar/
% https://www.fordfoundation.org/about/library/reports-and-studies/roads-and-bridges-the-unseen-labor-behind-our-digital-infrastructure/
it's hard to foresee or control which areas external contributors are changing.
At the beginning of the SA, some time was set aside to merge external
contributions (pull requests) which were already open. For the rest of the SA,
such contributions will be dealt with on a best effort basis, with the main
focus being this documentation and the work required for the plugin API.

% Vorgehen, Aufbau der Arbeit
\section{Methods and Structure}



% "Stand der Technik" (Was gibt es schon?)
\chapter{Existing Work}  

% Bestehende Lösungsansätze und Normen
\section{Existing Approaches and Norms (Existing APIs?)}

% Kurzbeschreibung und Charakterisierung
% \section{Summary and Characerization}

% Defizite (Hinweise auf Weiterentwicklungs-, bzw. Verbesserungspotential)
\section{Problems with existing solutions}  


% Bewertung (Evaluation)
\chapter{Evaluation}

% Kriterien (Wie wird bewertet?)
\section{Criteria}

% Schlussfolgerungen, eigener Lösungsansatz
\section{Conclusion}


% Umsetzungskonzept (des eigenen Lösungsansatzes)
\chapter{Concept}

% Grobe Beschreibung des eigenen Lösungskonzepts
\section{Summary}

% z.T. Wiederholung im Groben, z.T. Verweise auf Teil II-Kapitel
\section{...}


% Resultate, Bewerung und Ausblick
\chapter{Results}

% Zielerreichung
\section{Achievement of Objectives}

% Ausblick: Weiterentwicklung (nur wichtigste Punkte)
\section{Future Work}

% Persönliche Berichte
\section{Personal Review}

% Dank
\section{Thanks}



% Teil II SW-Projektdokumentation ("klassisches" RUP 2/HSR)  
\part{Project Documentation}


% Vision (ev. auf Kapitel 1.1 verweisen)
\chapter{Vision}


% Anforderungsspezifikation
\chapter{Requirements Specification}

% Anforderung an die Arbeit
\section{Requirements of the Project}

% Use Cases (Success Scenario / Success Diagram)
\section{Use Cases}

% System-Sequenzdiagramme
\section{System Sequence Diagrams}

% Weitere Funktionen, die nicht erfasst wurden
\section{Further Functionality}

% Nicht-funktionale Anforderungen (Rahmbenbedingungen, evtl. Verweis auf 1.3)
\section{Non-functional Requirements}

% Detailspezifikation
\section{Detailed Specification}


% Analyse (Business Model)
\chapter{Analysis}

% Domain Modell, Klassendiagramme (konzeptionell)
\section{Domain Model}

% Objektkatalog (Beschreibung der Konzepte, bzw. Entitätsmengen)
\section{Objects}


% Design (Entwurf)
\chapter{Design}

% Architektur
\section{Architecture}

% Objektkatalog (Klassenkonzepte, Verantwortlichkeiten und Konsistenzbedingungen)
\section{Objects}

% Package- und Klassendiagramme (konzeptionell)
\section{Packages and Classes}

% Sequenzdiagramm, UI Design
\section{Sequence Diagrams}


% Implementation (Entwicklung)
\chapter{Implementation and Test}

% Implementation: Erläuterungen wichtiger konkreter Klassen
\section{Important Classes}

% Automatische Testverfahren
\section{Automated Testing}

% Manuelle Testverfahren, etc.
\section{Manual Testing}


% Resultate und Weiterentwicklung
\chapter{Results and Future Work}

% Resultate (ev. nach oben in Teil I Kap. 5 eingleidern)
\section{Results}

% Möglichkeiten der Weiterentwicklung
\section{Possible Future Work}

% Vorgehen (welche Mögl. würde man nun wie weiterentwickeln?)
\section{Future Approach}


% Projektmanagement (Planung, Soll)
\chapter{Project Management}

% Prototypen, Releases, Meilensteine
\section{Prototypes, Releases, Milestone}

% Team, Rollen und Verantwortlichkeiten
\section{Team and Roles}
\emph{Florian Bruhin} is both the primary maintainer of qutebrowser and the
author of this student research project. He's been working on qutebrowser since
December 2013 and started studying Computer Science at HSR in 2016.

Professor \emph{Stefan Keller}, institute partner at the Institute for Software
(IFS) at HSR is the advisor for this project.

The qutebrowser community is not directly involved in this research project, but
is the primary audience of the resulting work. It has also contributed many
ideas and use cases for future
plugins\footnote{\url{https://github.com/qutebrowser/qutebrowser/issues/30}}.

\label{fiete}
\emph{Fritz Reichwald} (fiete201\footnote{\url{https://github.com/fiete201}})
is a long-time qutebrowser user who is working on migrating qutebrowser's
documentation system from asciidoc\footnote{\url{http://asciidoc.org/}} (which
is deprecated) to the Sphinx\footnote{\url{http://www.sphinx-doc.org/}}
documentation generator. This change was
planned\footnote{\url{https://github.com/qutebrowser/qutebrowser/issues/345}}
since December 2014, so his help with tackling this is very much appreciated.
This project will benefit from his work, as Sphinx, unlike asciidoc, is a
very good fit for documenting APIs. His work is clearly marked as such in this
documentation.

% Aufwandschätzung, Zeitplan, Projektplan
\section{Project Schedule}
% Risiken
\section{Risks}
% Prozessmodel
\section{Process Model}


% Projektmonitoring (Ist-Beschreibung, so ist es passiert)
\chapter{Project Monitoring}
% Soll-Ist-Zeit-Vergleich
\section{Allocated/Actual Time}
% Codestatistik (Zeilen: Kommentare, Klassen, Packages)
\section{Code Statistics}


\chapter{Software Documentation}
% Maybe separate (Sphinx)



\part{Appendices}


\chapter{Glossary and Abbreviations}

Test \citep{yapproach}

\renewcommand{\bibname}{\chapter{Literature and Sources}}
\bibliographystyle{IEEEtranN}
\bibliography{bibliography}

\end{document}