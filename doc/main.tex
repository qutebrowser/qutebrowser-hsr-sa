\documentclass[a4paper,parskip=full]{scrreprt}

\usepackage[T1]{fontenc}
\usepackage{cmbright}
\usepackage[utf8]{inputenc}
\usepackage{hyperref}
\usepackage{booktabs}
\usepackage{graphicx}

\begin{document}

\title{qutebrowser made extendible \\ FIXME}
\author{Florian Bruhin \\ \url{florian@qutebrowser.org}}
\date{\today}
\maketitle

\chapter*{Abstract}

\chapter*{Management Summary}


\tableofcontents
\listoffigures
\listoftables


% Aufgabenstellung
\chapter*{Task Definition}  

% Technischer Bericht
\part{Technical Report}

% Einführung
\chapter{Introduction}
% Problemstellung, Vision
\section{Vision}
% Ziele und Unterziele
\section{Goals}
% Rahmenbedingungen, Umfeld, Definitionen, Abgrenzungen
\section{Context}
% Vorgehen, Aufbau der Arbeit
\section{Methods and Structure}

% "Stand der Technik" (Was gibt es schon?)
\chapter{Existing Work}  
% Bestehende Lösungsansätze und Normen
\section{Existing Approaches and Norms (Existing APIs?)}
% Kurzbeschreibung und Charakterisierung
% \section{Summary and Characerization}
% Defizite (Hinweise auf Weiterentwicklungs-, bzw. Verbesserungspotential)
\section{Problems with existing solutions}  

% Bewertung (Evaluation)
\chapter{Evaluation}
% Kriterien (Wie wird bewertet?)
\section{Criteria}
% Schlussfolgerungen, eigener Lösungsansatz
\section{Conclusion}

% Umsetzungskonzept (des eigenen Lösungsansatzes)
\chapter{Concept}
% Grobe Beschreibung des eigenen Lösungskonzepts
\section{Summary}
% z.T. Wiederholung im Groben, z.T. Verweise auf Teil II-Kapitel

% Resultate, Bewerung und Ausblick
\chapter{Results}
% Zielerreichung
\section{Achievement of Objectives}
% Ausblick: Weiterentwicklung (nur wichtigste Punkte)
\section{Future Work}
% Persönliche Berichte
\section{Personal Review}
% Dank
\section{Thanks}


% Teil II SW-Projektdokumentation ("klassisches" RUP 2/HSR)  
\part{Project Documentation}
% Vision (ev. auf Kapitel 1.1 verweisen)
\chapter{Vision}

% Anforderungsspezifikation
\chapter{Requirements Specification}
% Anforderung an die Arbeit
\section{Requirements of the Project}
% Use Cases (Success Scenario / Success Diagram)
\section{Use Cases}
% System-Sequenzdiagramme
\section{System Sequence Diagrams}
% Weitere Funktionen, die nicht erfasst wurden
\section{Further Functionality}
% Nicht-funktionale Anforderungen (Rahmbenbedingungen, evtl. Verweis auf 1.3)
\section{Non-functional Requirements}
% Detailspezifikation
\section{Detailed Specification}

% Analyse (Business Model)
\chapter{Analysis}
% Domain Modell, Klassendiagramme (konzeptionell)
\section{Domain Model}
% Objektkatalog (Beschreibung der Konzepte, bzw. Entitätsmengen)
\section{Objects}

% Design (Entwurf)
\chapter{Design}
% Architektur
\section{Architecture}
% Objektkatalog (Klassenkonzepte, Verantwortlichkeiten und Konsistenzbedingungen)
\section{Objects}
% Package- und Klassendiagramme (konzeptionell)
\section{Packages and Classes}
% Sequenzdiagramm, UI Design
\section{Sequence Diagrams}

% Implementation (Entwicklung)
\chapter{Implementation and Test}
% Implementation: Erläuterungen wichtiger konkreter Klassen
\section{Important Classes}
% Automatische Testverfahren
\section{Automated Testing}
% Manuelle Testverfahren, etc.
\section{Manual Testing}

% Resultate und Weiterentwicklung
\chapter{Results and Future Work}
% Resultate (ev. nach oben in Teil I Kap. 5 eingleidern)
\section{Results}
% Möglichkeiten der Weiterentwicklung
\section{Possible Future Work}
% Vorgehen (welche Mögl. würde man nun wie weiterentwickeln?)
\section{Future Approach}

% Projektmanagement (Planung, Soll)
\chapter{Project Management}
% Prototypen, Releases, Meilensteine
\section{Prototypes, Releases, Milestone}
% Team, Rollen und Verantwortlichkeiten
\section{Team and Roles}
% Aufwandschätzung, Zeitplan, Projektplan
\section{Project Schedule}
% Risiken
\section{Risks}
% Prozessmodel
\section{Process Model}

% Projektmonitoring (Ist-Beschreibung, so ist es passiert)
\chapter{Project Monitoring}
% Soll-Ist-Zeit-Vergleich
\section{Allocated/Actual Time}
% Codestatistik (Zeilen: Kommentare, Klassen, Packages)
\section{Code Statistics}

\chapter{Software Documentation}
% Maybe separate (Sphinx)


\part{Appendices}

\chapter{Glossary and Abbreviations}

\chapter{Literature and Sources}



\end{document}