% -*- TeX-master: "main.tex" -*-

% Umsetzungskonzept (des eigenen Lösungsansatzes)
\chapter{Concept}
\label{ch:concept}

% Grobe Beschreibung des eigenen Lösungskonzepts
\section{Summary}
As shown in previous chapters, there are ample reasons for qutebrowser's
extension API to differ from existing approaches in other browsers. Much of the
functionality being used in extensions already exists as a cleanly defined API
in qutebrowser, which could be exposed (after minor refactorings) in the
extension API.

Due to time constraints, not all considerations made could find their way into
an API design. Instead, the work being done was centered on improving APIs which
already exist in qutebrowser, in order to separate existing code into extensions.

\section{Exposing existing APIs}
Many of the APIs which should be exposed to extensions already exist in
qutebrowser. However, exposing formerly private APIs carries certain risks:

\begin{itemize}
  \item Often, APIs do not undergo any review before they are made public
    \autocite[18f]{api-design}, This can lead to mistakes which were not regarded
    as important (or not noticed at all) to be exposed publicly, without a way
    to remedy the mistakes (since that would be an incompatible change).
  \item Looking at the tab API explained in section \ref{tabapi}, some methods
    exist because there was a need in the core part for them -- but they should
    not be exposed to plugins, since they are only needed for very specialized
    reasons.
  \item When improving or refactoring APIs in the future, care must be taken
    because third-party extensions might be affected by the change. To avoid
    extension authors having to catch up with every upgrade of the core (and
    frustrating users due to breaking extensions), any future changes should be
    backwards-compatible, which greatly limits the room for future cleanups.
  \item Even for methods which seem useful for an extension API, their need
    should be carefully evaluated by looking at proposals for future extensions
    before blindly exposing them. This approach leads to a smaller (and thus
    simpler) API. It also increases the maintainability of the core, as parts
    which aren't exposed can be freely changed without needing to take care of
    extensions.
\end{itemize}

It must be noted that many of those issues are less relevant in the scope of
this SA, as the API will be open to third-party extensions at a later point.
However, care was taken to mitigate those issues where possible.

In order to catch possible issues with existing APIs planning to be exposed to
extensions, the existing APIs were reviewed by Raphael~Das~Gupta from the IFS
(Institute for Software).

It was considered to implement the object adapter pattern \autocite[139]{gof} to
limit the methods exposed in the API to what is actually necessary:

\begin{figure}[h]
\centering
\begin{sequencediagram}
  \newinst{extension}{:Extension}
  \newinst{adapter}{:Adapter}
  \newinst{tab}{:Tab}
  \newinst{view}{:QWebEngineView}

  \begin{call}{extension}{title()}{adapter}{``Title''}
    \begin{call}{adapter}{title()}{tab}{``Title''}
      \begin{call}{tab}{title()}{view}{``Title''}
      \end{call}
    \end{call}
  \end{call}
\end{sequencediagram}
\caption{Sequence diagram for the extension API with adapter object}
\end{figure}

However, this would lead to a considerable amount of duplicated code: Big parts
of the tab API would need to be duplicated, just to pass calls through. Since
this solution provided an overhead which was considered too big compared to the
benefits, the following alternatives were considered:

\begin{itemize}
  \item The methods could be marked private via the common Python convention to
    start the name with an underscore (for example, \py{_shutdown()}).
    However, in the core part, those should \emph{not} be private, as they need to
    be used from outside the tab API.
  \item The methods could be lacking a Python docstring, so they are accessible,
    but excluded from the extension API documentation. However, auto-completion
    in development environments would still suggest those methods, which could
    lead to users thinking they are part of the exposed API. Additionally, this
    would make it impossible to document the affected methods for developers
    working on the core.
  \item Similar to existing members like \verb|.scroller| or \verb|.zoom|, a new
    \verb|.private_api| member could be introduced, which points to an object
    containing any methods not part of the public API.
\end{itemize}

The last solution was chosen since it clearly communicates (both in the
documentation and the name of the attribute) what part of the API is intended
for usage in the core only. This solution also still allows the core to access
those APIs, and the private API can still be documented properly.

\newenvironment{umlhighlight}{%
  \let\oldumlfillcolor\umlfillcolor%
  \renewcommand{\umlfillcolor}{yellow}%
}{%
  \renewcommand\umlfillcolor{\oldumlfillcolor}
}

\begin{figure}[h]
\begin{tikzpicture}
  \begin{umlhighlight}
    \begin{class}[text width=2cm]{TabPrivate}{2,2}
    \end{class}
    \mycomposition{Tab}{}{}{TabPrivate}{}{}
  \end{umlhighlight}

  \begin{class}[text width=2cm]{Tab}{6,2}
  \end{class}

  \begin{class}[text width=2cm]{Audio}{0,0}
  \end{class}
  \mycomposition{Tab}{}{}{Audio}{}{}

  \begin{class}[text width=2cm]{Elements}{1.5,4.5}
  \end{class}
  \mycomposition{Tab}{}{}{Elements}{}{}

  \begin{class}[text width=2cm]{History}{3,0}
  \end{class}
  \mycomposition{Tab}{}{}{History}{}{}

  \begin{umlhighlight}
  \begin{class}[text width=3cm]{HistoryPrivate}{3,-2}
  \end{class}
  \mycomposition{History}{}{}{HistoryPrivate}{}{}
  \end{umlhighlight}

  \begin{class}[text width=2cm]{Scroller}{4.5,4.5}
  \end{class}
  \mycomposition{Tab}{}{}{Scroller}{}{}

  \begin{class}[text width=2cm]{Caret}{6,0}
  \end{class}
  \mycomposition{Tab}{}{}{Caret}{}{}

  \begin{class}[text width=2cm]{Zoom}{7.5,4.5}
  \end{class}
  \mycomposition{Tab}{}{}{Zoom}{}{}

  \begin{class}[text width=2cm]{Search}{9,0}
  \end{class}
  \mycomposition{Tab}{}{}{Search}{}{}

  \begin{class}[text width=2cm]{Printing}{10.5,4.5}
  \end{class}
  \mycomposition{Tab}{}{}{Printing}{}{}

  \begin{class}[text width=2cm]{Action}{12,0}
  \end{class}
  \mycomposition{Tab}{}{}{Action}{}{}

  \begin{class}[text width=2cm]{TabData}{10,2}
  \end{class}
  \mycomposition{Tab}{}{}{TabData}{}{}
\end{tikzpicture}
\caption[Class diagram for refactored tab API]{Class diagram for refactored tab
  API, with added classes highlighted}
\end{figure}

The full refactored API can be found in the developer documentation in appendix
\ref{ch:sphinx}.

% z.T. Wiederholung im Groben, z.T. Verweise auf Teil II-Kapitel
\section{Security}
\label{security}
As part of this SA, third-party extensions are out of scope; only code which was
part of qutebrowser's core is moved into extensions. Thus, it seems like there
are no special security considerations to be made. However, the architecture of
an extension API is fundamentally influenced by such considerations, and support
for third-party extensions will be added in the near future. Therefore, the
security philosophy of qutebrowser's extensions is analyzed in this section.

The security model of WebExtensions (see section \ref{webextensions}) assumes
that extensions are untrusted. Even with the limited WebExtensions API,
malicious extensions are a common issue
\autocite{mozilla-signing,mozilla-trustworthy}. Browser vendors try to alleviate
this problem with automated and manual code review, extension signing, and
blacklisting of known-bad extensions. Additionally, the user explicitly needs to
allow extensions to run in incognito/private browsing mode, as the impact of a
privacy breach typically is considerably larger in that scenario.

The approach taken qutebrowser extensions is different: Extensions are treated
as trusted, so users are responsible for reviewing extensions before installing
them. This is for a variety of reasons:

\begin{itemize}
  \item Extensions for qutebrowser should be written in Python (like qutebrowser itself
    is), but safely executing untrusted Python code is commonly regarded to be
    impossible \autocite{nedbat-eval, lwn-pysandbox}.
  \item The attack surface for a malicious actor trying to distribute a bad
    extension is much smaller, since qutebrowser caters to a relatively small
    niche group of users. Thus, malicious extensions are expected to be a seldom
    occurrence compared to more common browsers such as Google Chrome or Mozilla
    Firefox.
  \item Users of qutebrowser are typically power users, due to its
    keyboard-focused nature. With a browser aimed at casual users, it might be
    easy to trick them into installing an extension which is malicious. In
    contrast, users of qutebrowser can be expected to be much more diligent in ensuring that
    the extensions they're installing are not malicious.
  \item The volume of available extensions is much lower. Thus, it's conceivable
    that there's a whitelist of extensions which have been reviewed and approved
    by one of qutebrowser's core developers. A contributor wishing to distribute
    a new extension could then ask for it to be reviewed and included in that list.
  \item Due to the focus on power users, a user should always be able to
    install an extension manually, or write a custom one. Thus, mandatory
    extension signing or approval by a central body is undesirable, as it
    presents a trade-off between a user's freedom and security.
\end{itemize}

% % Design (Entwurf)
% \chapter{Design}
% 
% % Architektur
% \section{Architecture}
% 
% % Objektkatalog (Klassenkonzepte, Verantwortlichkeiten und Konsistenzbedingungen)
% \section{Objects}
% 
% % Package- und Klassendiagramme (konzeptionell)
% \section{Packages and Classes}
% 
% % Sequenzdiagramm, UI Design
% \section{Sequence Diagrams}

% Implementation (Entwicklung)