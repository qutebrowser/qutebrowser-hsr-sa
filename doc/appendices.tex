% -*- TeX-master: "main.tex" -*-

\begin{appendices}
\chapter{Glossary and Abbreviations}
\begin{multicols}{2}
\label{ch:glossary}
\begin{description}[leftmargin=0pt]
  \item[API]{Application Programming Interface, i.e., in
      case of an extension API, the functions and classes (names, arguments,
      etc.) an extension can implement.}
  \item[add-on]{A synonym for \emph{extension}.}
  \item[backend]{The software library doing the ``heavy lifting'' in
      qutebrowser, such as doing network requests or drawing website content.}
  \item[CI]{Continuous integration, i.e., running automated tests or other
      checks with every change.}
  \item[CSS]{Cascading Style Sheets, used to define styling/appearance of web pages.}
  \item[decorator] A \emph{Python} decorator allows annotating/wrapping a
    function. See section \ref{sec:langfeatures}.
  \item[DOM]{Document Object Model, the \emph{API} to access a HTML document as
      a tree structure.}
  \item[extension]{Code using an \emph{API} in order to extend the functionality
      of an existing project.}
  \item[GUI]{Graphical User Interface}
  \item[HSR]{The Hochschule für Technik (University of Applied Sciences) in
      Rapperswil, Switzerland.}
  \item[HTML]{HyperText Markup Language, the language used to structure web
      pages.}
  \item[IDE]{Integrated Development Environment -- a software application
      providing tools (such as a source editor) for development.}
  \item[IFS]{The Institute for Software at \emph{HSR}.}
  \item[JavaScript]{The programming language used for interactive web sites.}
  \item[plugin]{In most contexts, the same as an \emph{extension}. Note that
      qutebrowser initially used \emph{plugin API} to refer to its extension
      API. However, this should be avoided, as ``plugin'' is too ambiguous in
      the context of web browsers: A plugin usually refers to software using the
      deprecated NPAPI (Netscape Plugin API) or PPAPI (Pepper Plugin API)
      technologies, such as Adobe Flash.}
  \item[Python]{The programming language used to write qutebrowser.}
  \item[PyQt]{A software library which allows to use \emph{Qt} from \emph{Python}.}
  \item[Qt]{The \emph{GUI} library used by qutebrowser}
  \item[QtWebEngine]{The \emph{backend} used by default in qutebrowser, based on
      the Chromium project (like the Chrome web browser).}
  \item[QtWebKit]{One of two possible \emph{backends} qutebrowser can use.
      QtWebKit is the older backend, which is not used by default anymore, but is
      still supported.}
  \item[SA]{Studienarbeit (Student research project)}
  \item[slot] A slot is the method or function connected to a \emph{signal} in
    \emph{Qt}. See section \ref{sec:langfeatures}.
  \item[signal] A signal is a \emph{Qt} feature which lets objects communicate
    with each other. See section \ref{sec:langfeatures}.
  \item[W3C]{World Wide Web Consortium, the standards body for web-related
      technologies such as HTML.}
\end{description}
\end{multicols}

\chapter{API documentation}
\label{ch:sphinx}
The API documentation on the following pages has been generated from Python
docstrings in the API implementation. Sphinx 1.8.2 was used.

{\let\clearpage\relax\chapter{Task description / Aufgabenstellung}}
The original task description for this project (written in German) can be found on page \pageref{ch:aufgabenstellung}ff.

\includepdf[pages={5-15,17-18},trim=0 2cm 0 1.8cm,clip,pagecommand=,offset=0 1cm]{./img/sphinx.pdf}
\label{ch:aufgabenstellung}
\includepdf[pages=-,pagecommand=]{./img/SA_Bruhin_Aufgabenstellung_v2.pdf}

\chapter{Test log}
\label{ch:testlog}
\begin{minted}{text}
========================== test session starts ==========================
platform linux -- Python 3.7.1, pytest-4.0.1, py-1.7.0, pluggy-0.8.0
cachedir: .tox/py37/.pytest_cache
PyQt5 5.11.3 -- Qt runtime 5.12.0 -- Qt compiled 5.12.0
benchmark: 3.1.1 (defaults: [...])
hypothesis profile 'default' -> [...]
rootdir: /home/florian/proj/qutebrowser/git, inifile: pytest.ini
plugins: xvfb-1.1.0, travis-fold-1.3.0, rerunfailures-5.0, repeat-0.7.0,
  qt-3.2.1, mock-1.10.0, instafail-0.4.0, faulthandler-1.5.0, cov-2.6.0,
  benchmark-3.1.1, bdd-3.0.0, hypothesis-3.82.5
collected 195 items

tests/unit/api/test_cmdutils.py ..........................................
                                ...................
tests/unit/components/test_adblock.py ..............................
tests/unit/components/test_misccommands.py .....
tests/unit/extensions/test_loader.py ............
tests/unit/browser/webengine/test_webenginetab.py .....
tests/end2end/features/test_caret_bdd.py ...........
tests/end2end/features/test_scroll_bdd.py ..................s.............
                                          ...................
tests/end2end/features/test_zoom_bdd.py ...............x....


--------------- benchmark: 1 tests ---------------
Name (time in us)             Min      Max  Median
--------------------------------------------------
test_adblock_benchmark     3.3040  40.3980  3.4670
--------------------------------------------------

Legend:
  Outliers: 1 Standard Deviation from Mean; 1.5 IQR (InterQuartile Range)
            from 1st Quartile and 3rd Quartile.
  OPS: Operations Per Second, computed as 1 / Mean
=========== 193 passed, 1 skipped, 1 xfailed in 44.72 seconds ===========
\end{minted}

Tests marked with \emph{s}/\emph{skipped} were skipped because of
platform/environment differences. Tests marked with \emph{x}/\emph{xfailed} were
expected to fail due to known bugs.

Note only the subset of tests relevant to this project was ran.

\renewcommand{\bibname}{\chapter{Literature and Sources}\vspace{-1em}}
{\let\clearpage\relax\printbibliography}
\end{appendices}
